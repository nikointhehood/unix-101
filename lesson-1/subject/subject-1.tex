 \documentclass[12pt]{article}
%\usepackage[portuguese]{babel}
\usepackage{natbib}
\usepackage{url}
\usepackage[utf8x]{inputenc}
\usepackage{amsmath}
\usepackage{graphicx}
\graphicspath{{images/}}
\usepackage{parskip}
\usepackage{fancyhdr}
\usepackage{vmargin}
\setmarginsrb{1.5 cm}{2.5 cm}{1.5 cm}{2.5 cm}{1 cm}{1.5 cm}{1 cm}{1.5 cm}
\usepackage{hyperref}
\hypersetup{
    colorlinks=true,
    citecolor=black,
    filecolor=black,
    linkcolor=black,
    urlcolor=black
}
\let\oldhref\href
\renewcommand{\href}[2]{\oldhref{#1}{\bfseries#2}}

\usepackage[bottom]{footmisc}
\usepackage{caption}
\DeclareCaptionFormat{sanslabel}{#3}%
\usepackage[section]{placeins}
\usepackage{xcolor}
\definecolor{light-gray}{gray}{0.95}
\usepackage{listings}
\lstset{basicstyle=\ttfamily,
  showstringspaces=false,
  commentstyle=\color{red},
  keywordstyle=\color{blue},
  backgroundcolor=\color{light-gray},
  breaklines=true,
  extendedchars=true,
  literate={é}{{\'e}}1
}
\lstset{aboveskip=15pt,belowskip=15pt}
\usepackage[autostyle]{csquotes}  
\def\labelitemi{--}
\usepackage{enumitem}
\setlist{nosep}
\usepackage{booktabs}
\usepackage{datetime}

% Quoting magic to have the Author of the quote after the actual quote
\let\oldquote\quote
\let\endoldquote\endquote
\renewenvironment{quote}[2][]
  {\if\relax\detokenize{#1}\relax
     \def\quoteauthor{#2}%
   \else
     \def\quoteauthor{#2~---~#1}%
   \fi
   \oldquote}
  {\par\nobreak\smallskip\hfill(\quoteauthor)%
   \endoldquote\addvspace{\bigskipamount}}


\title{Subject 1}								% Title
%\author{}								% Author
\date{\today}											% Date

\makeatletter
\let\thetitle\@title
%\let\theauthor\@author
\let\thedate\@date
\makeatother

\pagestyle{fancy}
\fancyhf{}
%\rhead{\theauthor}
\lhead{\thetitle}
\cfoot{\thepage}

\begin{document}

%%%%%%%%%%%%%%%%%%%%%%%%%%%%%%%%%%%%%%%%%%%%%%%%%%%%%%%%%%%%%%%%%%%%%%%%%%%%%%%%%%%%%%%%%

\begin{titlepage}
	\centering
    \vspace*{0.5 cm}
    \includegraphics[scale = 0.3]{logo4.png}\\[1.0 cm]
    \textsc{\LARGE \newline\newline UNIX 101}\\[2.0 cm]
	\textsc{\Large or "How to feel like a true hack3r"}\\[0.5 cm]
	\rule{\linewidth}{0.2 mm} \\[0.4 cm]
	{ \huge \bfseries \thetitle}\\
	\rule{\linewidth}{0.2 mm} \\[1.5 cm]
	
    \thedate
    
    
    
	
\end{titlepage}

%%%%%%%%%%%%%%%%%%%%%%%%%%%%%%%%%%%%%%%%%%%%%%%%%%%%%%%%%%%%%%%%%%%%%%%%%%%%%%%%%%%%%%%%%

\tableofcontents
\pagebreak

%%%%%%%%%%%%%%%%%%%%%%%%%%%%%%%%%%%%%%%%%%%%%%%%%%%%%%%%%%%%%%%%%%%%%%%%%%%%%%%%%%%%%%%%%

\section{Foreword}
\subsection{Notions seen in the tutorial}

Welcome to subject-1! If you have finished tutorial-1 (and I hope you do, if you have not yet, please finish it now), you should know:

\begin{itemize}
\item What the syntax of a command is
\item A few useful commands
\item How to navigate in a filesystem
\item How to create, edit and remove files
\item What is a packet manager and how to use one 
\item How to create your own executable scripts
\end{itemize}

This is just enough to begin writing code in a local environment.

\subsection{Objectives}

The goal of this subject is to make you use the notions seen in the tutorial through coding exercises.
Additionally, we will have a look over key concepts to development and UNIX in general.

Last but not least, you are going to be evaluated on the exercises of this document. You are not expected to finish them during the course: you have an additional two weeks to work on them before having to hand in your solutions.

\section{Setup}
\subsection{A word on git}

\begin{quote}{\texttt{man git}}
\textit{Git is a fast, scalable, distributed \textbf{revision control system} with and unusually rich command set that provides both high-level operations and full access to internals.}
\end{quote}

To explain it in clearer terms, \texttt{git} is a tool that allows one to versionate documents. In software engineering (actually, in all of the fields of computer engineering), it is primordial to keep track of its work, especially when collaborating on the same projects.

\begin{itemize}
\item You have been working alone on a project for two months but one day you mess things up and delete some files? \texttt{git} can rollback the changes.
\item You want to try to revamp your code but are not sure that you will be able to have a working version on time for the deadline? With \texttt{git}, you can keep a marker on the last working version and can rollback to it at any time.
\item You are working in a team on the same project and want to share your work without conflicts? A \texttt{git} server can save your life.
\end{itemize}

Every serious company in the world uses a revision control system and fast, scalable and distributed attributes, \texttt{git} is by far the most popular.

\subsection{Installation}

We are not going to go into the details on \texttt{git}'s usage and huge set of features as it would take some time, but we are going to use it to download resources for the exercises.

First of all, install git if you do not already have it installed in your system. You should know how to install a package by now, if not, read again section 4 of the tutorial!

\section{Example}

\section{Exercise}
\subsection{Exo uno}
\subsection{Exo duo}

\section{Evaluation}
\subsection{Instructions}
\newdate{duedate}{30}{01}{2019}

You have to send a tarball (a \texttt{.tar}, not a \texttt{.zip} nor a \texttt{.rar}) of your work before \displaydate{duedate}, 23h59 at nikointhehood@gmail.com.

The subject and the content of your email are not important but your tarball should follow the format \texttt{firstname\_lastname.tar.bz2} (of course, you have to replace \texttt{firstname} and \texttt{lastname} by yours).

The command to work with tarballs is \texttt{tar}. To create one, use the following command:

\begin{lstlisting}[language=bash]
$ls
subject/ tutorial/
# c stands for create, v stands for verbose, f allows one to input a custom name for the archive and j indicates the usage of bzip2
$tar cvfj firstname_lastname.tar.bz2 subject/
(...)
$ls
subject/ tutorial firstname_lastname.tar.bz2
\end{lstlisting}

To verify that your archive contains everything you want, copy it somewhere else and decompress it:

\begin{lstlisting}[language=bash]
$ls
subject/ tutorial firstname_lastname.tar.bz2
$mkdir /tmp/test
$cp firstname_lastname.tar.bz2 /tmp/test
$cd /tmp/test
# x stands for eXtract, v for verbose and f allows one to indicate which file tar should work with
$tar xvf firstname_lastname.tar.bz2
$ls -l
# You should see what I will obtain when extracting your archive
(...)
\end{lstlisting}

\subsection{A word on cheating}

You have to remember that you should be studying for your own good. Cheating will not bring you any good in the long term; it is fine not to be able to finish every exercise of the subject, your main goal is to train and learn things.

Any form of cheating will immediately bring your grade down to 0. Additionally, your main teacher will be taken notice of that.

\textit{Note:} Changing the name of some variables will of course not trick the anti-cheat engine :)

\end{document}