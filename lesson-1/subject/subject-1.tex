 \documentclass[12pt]{article}
%\usepackage[portuguese]{babel}
\usepackage{natbib}
\usepackage{url}
\usepackage[utf8x]{inputenc}
\usepackage{amsmath}
\usepackage{graphicx}
\graphicspath{{images/}}
\usepackage{parskip}
\usepackage{fancyhdr}
\usepackage{vmargin}
\setmarginsrb{1.5 cm}{2.5 cm}{1.5 cm}{2.5 cm}{1 cm}{1.5 cm}{1 cm}{1.5 cm}
\usepackage{hyperref}
\hypersetup{
    colorlinks=true,
    citecolor=black,
    filecolor=black,
    linkcolor=black,
    urlcolor=black
}
\let\oldhref\href
\renewcommand{\href}[2]{\oldhref{#1}{\bfseries#2}}

\usepackage[bottom]{footmisc}
\usepackage{caption}
\DeclareCaptionFormat{sanslabel}{#3}%
\usepackage[section]{placeins}
\usepackage{xcolor}
\definecolor{light-gray}{gray}{0.95}
\usepackage{listings}
\lstset{basicstyle=\ttfamily,
  showstringspaces=false,
  commentstyle=\color{red},
  keywordstyle=\color{blue},
  backgroundcolor=\color{light-gray},
  breaklines=true,
  extendedchars=true,
  literate={é}{{\'e}}1
}
\lstset{aboveskip=15pt,belowskip=15pt}
\usepackage[autostyle]{csquotes}  
\def\labelitemi{--}
\usepackage{enumitem}
\setlist{nosep}
\usepackage{booktabs}
\usepackage{datetime}

\title{Subject 1}								% Title
%\author{}								% Author
\date{\today}											% Date

\makeatletter
\let\thetitle\@title
%\let\theauthor\@author
\let\thedate\@date
\makeatother

\pagestyle{fancy}
\fancyhf{}
%\rhead{\theauthor}
\lhead{\thetitle}
\cfoot{\thepage}

\begin{document}

%%%%%%%%%%%%%%%%%%%%%%%%%%%%%%%%%%%%%%%%%%%%%%%%%%%%%%%%%%%%%%%%%%%%%%%%%%%%%%%%%%%%%%%%%

\begin{titlepage}
	\centering
    \vspace*{0.5 cm}
    \includegraphics[scale = 0.3]{logo4.png}\\[1.0 cm]
    \textsc{\LARGE \newline\newline UNIX 101}\\[2.0 cm]
	\textsc{\Large or "How to feel like a true hack3r"}\\[0.5 cm]
	\rule{\linewidth}{0.2 mm} \\[0.4 cm]
	{ \huge \bfseries \thetitle}\\
	\rule{\linewidth}{0.2 mm} \\[1.5 cm]
	
    \thedate
    
    
    
	
\end{titlepage}

%%%%%%%%%%%%%%%%%%%%%%%%%%%%%%%%%%%%%%%%%%%%%%%%%%%%%%%%%%%%%%%%%%%%%%%%%%%%%%%%%%%%%%%%%

\tableofcontents
\pagebreak

%%%%%%%%%%%%%%%%%%%%%%%%%%%%%%%%%%%%%%%%%%%%%%%%%%%%%%%%%%%%%%%%%%%%%%%%%%%%%%%%%%%%%%%%%

\section{Foreword}
\subsection{Notions seen in the tutorial}
\subsection{Objectives}

\section{Setup}
\subsection{A word on git}
\subsection{Installation}

\section{Example}

\section{Exercise}
\subsection{Exo uno}
\subsection{Exo duo}

\section{Evaluation}

\subsection{Instructions}
\newdate{duedate}{30}{01}{2019}

You have to send a tarball (a \texttt{.tar}, not a \texttt{.zip} nor a \texttt{.rar}) of your work before \displaydate{duedate}, 23h59 at nikointhehood@gmail.com.

The subject and the content of your email are not important but your tarball should follow the format \texttt{firstname\_lastname.tar.bz2} (of course, you have to replace \texttt{firstname} and \texttt{lastname} by yours).

The command to work with tarballs is \texttt{tar}. To create one, use the following command:

\begin{lstlisting}[language=bash]
$ls
subject/ tutorial/
# c stands for create, v stands for verbose, f allows one to input a custom name for the archive and j indicates the usage of bzip2
$tar cvfj firstname_lastname.tar.bz2 subject/
(...)
$ls
subject/ tutorial firstname_lastname.tar.bz2
\end{lstlisting}

To verify that your archive contains everything you want, copy it somewhere else and decompress it:

\begin{lstlisting}[language=bash]
$ls
subject/ tutorial firstname_lastname.tar.bz2
$mkdir /tmp/test
$cp firstname_lastname.tar.bz2 /tmp/test
$cd /tmp/test
# x stands for eXtract, v for verbose and f allows one to indicate which file tar should work with
$tar xvf firstname_lastname.tar.bz2
$ls -l
# You should see what I will obtain when extracting your archive
(...)
\end{lstlisting}

\subsection{A word on cheating}

You have to remember that you should be studying for your own good. Cheating will not bring you any good in the long term.

Any form of cheating will immediately bring your grade down to 0. Additionally, your main teacher will be taken notice of that.

\textit{Note:} Changing the name of some variables will of course not trick the anti-cheat engine :)

\end{document}






