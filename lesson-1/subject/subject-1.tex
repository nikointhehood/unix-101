 \documentclass[12pt]{article}
%\usepackage[portuguese]{babel}
\usepackage{natbib}
\usepackage[hyphens]{url}
\usepackage[utf8x]{inputenc}
\usepackage{amsmath}
\usepackage{graphicx}
\graphicspath{{images/}}
\usepackage{parskip}
\usepackage{fancyhdr}
\usepackage{vmargin}
\setmarginsrb{1.5 cm}{2.5 cm}{1.5 cm}{2.5 cm}{1 cm}{1.5 cm}{1 cm}{1.5 cm}
\usepackage{hyperref}
\hypersetup{
    colorlinks=true,
    citecolor=black,
    filecolor=black,
    linkcolor=black,
    urlcolor=black
}
\let\oldhref\href
\renewcommand{\href}[2]{\oldhref{#1}{\bfseries#2}}

\usepackage[bottom]{footmisc}
\usepackage{caption}
\DeclareCaptionFormat{sanslabel}{#3}%
\usepackage[section]{placeins}
\usepackage{xcolor}
\definecolor{light-gray}{gray}{0.95}
\definecolor{medium-gray}{gray}{0.5}
\usepackage{listings}
\lstset{basicstyle=\ttfamily,
  showstringspaces=false,
  commentstyle=\color{red},
  keywordstyle=\color{blue},
  backgroundcolor=\color{light-gray},
  breaklines=true,
  extendedchars=true,
  literate={é}{{\'e}}1
}
\lstset{aboveskip=15pt,belowskip=15pt}
\usepackage[autostyle]{csquotes}  
\def\labelitemi{--}
\usepackage{enumitem}
\setlist{nosep}
\usepackage{booktabs}
\usepackage{datetime}
\lstdefinestyle{codestyle}{
    numbers=left,
    numbersep=10pt,
    numberstyle=\color{medium-gray}
}

% Quoting magic to have the Author of the quote after the actual quote
\let\oldquote\quote
\let\endoldquote\endquote
\renewenvironment{quote}[2][]
  {\if\relax\detokenize{#1}\relax
     \def\quoteauthor{#2}%
   \else
     \def\quoteauthor{#2~---~#1}%
   \fi
   \oldquote}
  {\par\nobreak\smallskip\hfill(\quoteauthor)%
   \endoldquote\addvspace{\bigskipamount}}


\title{Subject 1}								% Title
%\author{}								% Author
\date{\today}											% Date

\makeatletter
\let\thetitle\@title
%\let\theauthor\@author
\let\thedate\@date
\makeatother

\pagestyle{fancy}
\fancyhf{}
%\rhead{\theauthor}
\lhead{\thetitle}
\cfoot{\thepage}

\begin{document}

%%%%%%%%%%%%%%%%%%%%%%%%%%%%%%%%%%%%%%%%%%%%%%%%%%%%%%%%%%%%%%%%%%%%%%%%%%%%%%%%%%%%%%%%%

\begin{titlepage}
	\centering
    \vspace*{0.5 cm}
    \includegraphics[scale = 0.3]{logo4.png}\\[1.0 cm]
    \textsc{\LARGE \newline\newline UNIX 101}\\[2.0 cm]
	\textsc{\Large or "How to feel like a true hack3r"}\\[0.5 cm]
	\rule{\linewidth}{0.2 mm} \\[0.4 cm]
	{ \huge \bfseries \thetitle}\\
	\rule{\linewidth}{0.2 mm} \\[1.5 cm]
	
    \thedate
    
    
    
	
\end{titlepage}

%%%%%%%%%%%%%%%%%%%%%%%%%%%%%%%%%%%%%%%%%%%%%%%%%%%%%%%%%%%%%%%%%%%%%%%%%%%%%%%%%%%%%%%%%

\tableofcontents
\pagebreak

%%%%%%%%%%%%%%%%%%%%%%%%%%%%%%%%%%%%%%%%%%%%%%%%%%%%%%%%%%%%%%%%%%%%%%%%%%%%%%%%%%%%%%%%%

\section{Foreword}
\subsection{Notions seen in the tutorial}

Welcome to subject-1! If you have finished tutorial-1 (and I hope you do, if you have not yet, please finish it now), you should know:

\begin{itemize}
\item What the syntax of a command is
\item A few useful commands
\item How to navigate in a filesystem
\item How to create, edit and remove files
\item What is a packet manager and how to use one 
\item How to create your own executable scripts
\end{itemize}

This is just enough to begin writing code in a local environment.

\subsection{Objectives}

The goal of this subject is to make you use the notions seen in the tutorial through coding exercises.
Additionally, we will have a look over key concepts to development and UNIX in general.

Last but not least, you are going to be evaluated on the exercises of this document. You are not expected to finish them during the course: you have an additional two weeks to work on them before having to hand in your solutions.

\section{Setup}
\subsection{A word on git}

\begin{quote}{\texttt{man git}}
\textit{Git is a fast, scalable, distributed \textbf{revision control system} with and unusually rich command set that provides both high-level operations and full access to internals.}
\end{quote}

To explain it in clearer terms, \texttt{git} is a tool that allows one to versionate documents. In software engineering (actually, in all of the fields of computer engineering), it is primordial to keep track of its work, especially when collaborating on the same projects.

\begin{itemize}
\item You have been working alone on a project for two months but one day you mess things up and delete some files? \texttt{git} can rollback the changes.
\item You want to try to revamp your code but are not sure that you will be able to have a working version on time for the deadline? With \texttt{git}, you can keep a marker on the last working version and can rollback to it at any time.
\item You are working in a team on the same project and want to share your work without conflicts? A \texttt{git} server can save your life.
\end{itemize}

Every serious company in the world uses a revision control system and fast, scalable and distributed attributes, \texttt{git} is by far the most popular.

\subsection{Installation}

We are not going to go into the details on \texttt{git}'s huge set of features as it would take some time, but we are going to at least use it to download resources for the exercises.

First of all, install \texttt{git} if you do not already have it somewhere in your system. You should know how to install a package by now, if not, read again section 4 of the tutorial!

To check out how it works, let us download the git repository\footnote{A git repository is a directory managed by git} containing the exercises. The Github page is available here: \url{https://github.com/nikointhehood/unix-101}. In git language, we \textit{clone} a repository. So, let's clone it! Click on "Clone or download" and retrieve the clone url : \texttt{https://github.com/nikointhehood/unix-101.git}. Then, invoke the command \texttt{git clone}.

\begin{lstlisting}[language=bash]
$ls # The current directory is empty, let  us clone the repo unix-101
$git clone https://github.com/nikointhehood/unix-101.git
Cloning into 'unix-101'...
remote: Enumerating objects: 40, done.
remote: Counting objects: 100% (40/40), done.
remote: Compressing objects: 100% (23/23), done.
remote: Total 40 (delta 4), reused 40 (delta 4), pack-reused 0
Unpacking objects: 100% (40/40), done.
Checking connectivity... done.
$ls
unix-101/ # There we have it!
$ls unix-101/subject/
\end{lstlisting}

Github is a service hosting git servers. There exist others (Gitlab, Bitbucket, even private instances of Gitlab/Github/Bitbucket for companies, etc) but Github is widely used for open-source projects.
People from all over the world collaborate into writing software and sharing them to the world, for free. This kind of project is called \textit{open-source}: anybody can participate and add new features to them by suggesting changes to the code.
Actually, a lot of the most impressive computer engineering projects, extensively used by companies all over the world are open-source. The Linux kernel is findable on Github, for example\footnote{Here: \url{https://github.com/torvalds/linux}. Anybody can contribute, but that necessites some mad skillz.}.

\section{Exercise 0: max (example)}

Each exercise you are going to have to do is bundled with tests.
Let us proceed with an example.

\subsection{Goal}

\begin{description}
        \item[Script name:] \texttt{biggest.py}
        \item[Online help:] Check out the following links for more documentation
\begin{itemize}
	\item Command line arguments: \href{https://docs.python.org/3.0/library/sys.html\#sys.argv}{Python3 official documentation}
	\item Type conversion: \href{https://en.wikibooks.org/wiki/Python_Programming/Data_Types\#Type_conversion}{Wikibooks examples}
\end{itemize}
\end{description}

You have to write a Python3 script that will take two integer arguments and print the bigger one, followed by a line feed\footnote{In UNIX, a line feed is represented by a \texttt{\char`\\
n}. Some printing functions, like Python's \texttt{print()} automatically add one at the end of the printed message}.

If the two integers are equal, it should print \texttt{"The integers are equal"}, followed by a line feed

In case of an error, it should print \texttt{"Usage: ./biggest.py integer1 integer2"}, followed by a line feed.

Also, you are asked to provide a small file named \texttt{README} explaining how you solved the exercise.

The error cases are:

\begin{itemize}
	\item If there are less than two arguments
	\item If there are more than two arguments
	\item If the arguments are not integers
\end{itemize}

\subsection{Example}

\begin{lstlisting}[language=bash]
$./biggest.py 7 121
121
$./biggest.py -1 31
31
$./biggest.py 8
Usage: ./biggest.py integer1 integer2
$./biggest.py 1 2 3 4 5
Usage: ./biggest.py integer1 integer2
$./biggest.py 0.7 11
Usage: ./biggest.py integer1 integer2
\end{lstlisting}

\subsection{New notions}

To complete this exercise, you need to know a few more things.

\subsubsection{Arguments retrieval}

UNIX programs are launched through the command line. As we have seen with the many commands we learnt in the tutorial, one can pass command line arguments to change their behavior.

So how does a program retrieve the user's arguments? Well it is system-dependant but usually, programs can access them using the \texttt{argv} variable. In Python, one must first import the \texttt{sys} package and can then look for the program arguments in the variable \texttt{sys.argv}.

\texttt{sys.argv} is a list containing the following items:

\begin{itemize}
\item At index 0, the name of the Python program
\item At index 1, the first argument
\item At index 2, the second argument
\item At index 3, the third argument
\item ...
\end{itemize}

Let us check out an example:

\begin{lstlisting}[language=python,style=codestyle,title=print\_argv.py]
#! /usr/bin/python3

import sys
print("argv is a list of", len(sys.argv), "elements:", sys.argv)
\end{lstlisting}

Now, let us see what happens with zero, one or multiple command line arguments:

\begin{lstlisting}[language=bash]
$./print_argv.py 
argv is a list of 1 elements: ['./print_argv.py']
$./print_argv.py hello
argv is a list of 2 elements: ['./print_argv.py', 'hello']
$./print_argv.py hello 47
argv is a list of 3 elements: ['./print_argv.py', 'hello', '47']
$./print_argv.py hello 47 again?
argv is a list of 4 elements: ['./print_argv.py', 'hello', '47', 'again?']
$./print_argv.py hello 47 again? last_one333KK4..
argv is a list of 5 elements: ['./print_argv.py', 'hello', '47', 'again?', 'last_one333KK4..']
\end{lstlisting}

\subsubsection{Type casting}

There is an important concept in programmation: type casting. It allows one to change the type of a variable. The possible type of casts differ from one language to another.

It is extra useful because some functions only take some kind of types as argument.

Let us see some examples.

\begin{lstlisting}[language=python,style=codestyle,title=example\_cast.py]
#! /usr/bin/python3

my_str = "37.3"
print(my_str, "->", type(my_str))
# Some casts are possible:
print(list(my_str), "->", type(list(my_str)))
print(float(my_str), "->", type(float(my_str)))

# Some are impossible:
print(int(my_str), "->", type(int(my_str)))
\end{lstlisting}

Executing it, we see that we can make a \texttt{str} become a \texttt{float} or a \texttt{list} but not an \texttt{int}:

\begin{lstlisting}[language=bash]
37.3 -> <class 'str'>
['3', '7', '.', '3'] -> <class 'list'>
37.3 -> <class 'float'>
Traceback (most recent call last):
  File "./example_cast.py", line 10, in <module>
    print(int(my_str), "->", type(int(my_str)))
ValueError: invalid literal for int() with base 10: '37.3'
\end{lstlisting}

Actually, the error lies in the fact that the \texttt{37.3} can be interpreted by python as a float because it looks like one, but not as an \texttt{int} because \texttt{int} have round values.

Casting a \texttt{str} with a round value works:

\begin{lstlisting}[language=python,style=codestyle,title=example\_cast\_again.py]
#! /usr/bin/python3

my_str = "37"
print(my_str, "->", type(my_str))
# list, float and int should work with this value:
print(list(my_str), "->", type(list(my_str)))
print(float(my_str), "->", type(float(my_str)))
print(int(my_str), "->", type(int(my_str)))
\end{lstlisting}

Let us try:

\begin{lstlisting}[language=bash]
./example_cast_again.py 
37 -> <class 'str'>
['3', '7'] -> <class 'list'>
37.0 -> <class 'float'>
37 -> <class 'int'>
\end{lstlisting}


\subsection{Walkthrough}

Now this part will not always be there but I am going to guide you on how to solve exercise 0:

\subsubsection{Getting started}
Navigate to the directory \texttt{exercise-0}:

\begin{lstlisting}[language=bash]
$cd lesson-1/subject/exercises/exercise-0/
$ls
run_tests.py*
\end{lstlisting}

Here, there is only one executable script. It allows you to run the tests. Let us try:

\begin{lstlisting}[language=bash]
$./run_tests.py 
Test ['1', '2'] - FAILED
--> Your script encountered an error: File not found. Make sure your script is correctly named
\end{lstlisting}

Oh, an error. Well, that makes sense, we did not provide our solution to the exercise. So, let us solve it then.

\subsubsection{Working on the exercise}

Let's create a new file named \texttt{biggest.py} and write our solution.

\begin{lstlisting}[language=python,style=codestyle,title=biggest.py]
#! /usr/bin/python3
import sys # This import allows us to interact with the command line arguments

# First, we declare a function which will do the comparison job
def biggest(int1, int2):
    if int1 > int2:
        print(int1)
    elif int2 > int1:
        print(int2)
    else:
        print("The integers are equal")

# Let's get the command line arguments
first_int = int(sys.argv[1])
second_int = int(sys.argv[2])

# Now, call the function with the arguments previously retrieved
biggest(first_int, second_int)
\end{lstlisting}

Now that the code seems good, let us test it manually.

\begin{lstlisting}[language=bash]
$chmod +x biggest.py
$./biggest.py 20 3
20
$./biggest.py 7 11
11
$./biggest.py -1 7
7
\end{lstlisting}

It looks fine, but is that enough? Let us launch the tests.

\begin{lstlisting}[language=bash]
$./run_tests.py
Test ['1', '2'] - PASSED
Test ['1123123', '2'] - PASSED
Test ['-17', '13'] - PASSED
Test ['0.3', '13'] - FAILED
--> Your script encountered an error: 
Traceback (most recent call last):
  File "./biggest.py", line 14, in <module>
    first_int = int(sys.argv[1])
ValueError: invalid literal for int() with base 10: '0.3'
\end{lstlisting}

It seems that our script is not doing well with edge cases. This test just demonstrated that floating (e.g 1.7 or -3.1) numbers generate a \texttt{ValueError} in our code.
The mission now is to handle all of the cases that we can think of and fix our script little by little.

To handle this error case, we simply have to check that the casts of the arguments to \texttt{int} do not raise a \texttt{ValueError}:

\begin{lstlisting}[language=python,style=codestyle,title=biggest.py]
#! /usr/bin/python3
import sys # This import allows us to access argv, the list containing the command line arguments

# First, we declare a function which will do the comparison job
def biggest(int1, int2):
    if int1 > int2:
        print(int1)
    elif int2 > int1:
        print(int2)
    else:
        print("The integers are equal")

# Let's try to get the command line arguments
try:
    first_int = int(sys.argv[1])
    second_int = int(sys.argv[2])
    # Now, call the function with the arguments previously retrieved
    biggest(first_int, second_int)
except ValueError:
    print("Usage: ./biggest.py integer1 integer2")
\end{lstlisting}

Running the tests again, we encounter another error on the number of arguments: this can be corrected by checking first that we are only given two additional arguments (the first one being the script filename, \texttt{biggest.py}).

\begin{lstlisting}[style=codestyle,language=python,title=biggest.py]
#! /usr/bin/python3
import sys # This import allows us to access argv, the list containing the command line arguments

# First, we declare a function which will do the comparison job
def biggest(int1, int2):
    if int1 > int2:
        print(int1)
    elif int2 > int1:
        print(int2)
    else:
        print("The integers are equal")

# Let's try to get the command line arguments
try:
    if len(sys.argv) != 3: # Actually, the target value is 3 because the first element of argv is always the name of the script
        print("Usage: ./biggest.py integer1 integer2")
    else:
        first_int = int(sys.argv[1])
        second_int = int(sys.argv[2])
        # Now, call the function with the arguments previously retrieved
        biggest(first_int, second_int)
except ValueError:
    print("Usage: ./biggest.py integer1 integer2")
\end{lstlisting}

Running the tests one last time will only display green success messages:
\begin{lstlisting}[language=bash]
./run_tests.py 
Test ['1', '2'] - PASSED
Test ['1123123', '2'] - PASSED
Test ['-17', '13'] - PASSED
Test ['0.3', '13'] - PASSED
Test ['13'] - PASSED
Test ['0.1'] - PASSED
Test ['1', '2', '3', '-8'] - PASSED
Test ['mdr', '3'] - PASSED
Test ['0', '3'] - PASSED
Test ['0.7', '1'] - PASSED
Test ['19', '19'] - PASSED
Test ['0.7', '0.3'] - PASSED

All tests passed! Good job :). Make sure you did not forget any untested behavior
\end{lstlisting}

This exercise is completed. As asked in the exercise, we should now write a small text in a file named \texttt{README} explaining how our algorithm works and what were the pain points:

\begin{lstlisting}[language=bash]
$ls
biggest.py*  README  run_tests.py*
$cat README
The goal of the exercise was to print the highest value of the command-line arguments.
I had difficulties with sanitizing the user input and I solved it using type casting and by checking the number of arguments in argv.
\end{lstlisting}

This kind of development mindset is the one you should follow for the next exercises.

\section{Exercise 1: max advanced}

\subsection{Goal}

\begin{description}
        \item[Script name:] \texttt{biggest\_advanced.py}
        \item[Online help:] Check out the following links for more documentation
\begin{itemize}
	\item Command line arguments: \href{https://docs.python.org/3.0/library/sys.html\#sys.argv}{Python3 official documentation}
	\item Type conversion: \href{https://en.wikibooks.org/wiki/Python_Programming/Data_Types\#Type_conversion}{Wikibooks examples}
\end{itemize}
\end{description}
      	
You have to write a Python3 script that will take an arbitrary number of integer arguments and print the bigger one, followed by a line feed.

In the case where all of the integers are the same, it should print one of them, followed by a line feed.

In case of an error, it should print:

\texttt{"Usage: ./biggest\_advanced.py integer1 integer2 [integer3]..."}, followed by a line feed.

Also, you are asked to provide a small file named \texttt{README} explaining how you solved the exercise.

The error cases are:

\begin{itemize}
	\item If there are less than two arguments
	\item If the arguments are not integers
\end{itemize}

\subsection{Example}

\begin{lstlisting}[language=bash]
$./biggest_advanced.py 3 17
17
./biggest_advanced.py 3 17 141
141
./biggest_advanced.py -1 77 -31 55
77
./biggest_advanced.py 3
Usage: ./biggest_advanced.py integer1 integer2 [integer3]...
./biggest_advanced.py 27 27 27
27
\end{lstlisting}


\section{Exercise 2: Long long words}
\subsection{Goal}

\begin{description}
        \item[Script name:] \texttt{longest\_words.py}
\end{description}

You have to write a Python3 script that will take an arbitrary number of \texttt{str} arguments and print the longest ones, separated by a comma,  in the order in which they appeared, followed by a line feed.

In case of an error, it should print:

\texttt{"Usage: ./longest\_word.py arg1 [arg2]..."}, followed by a line feed.

Also, you are asked to provide a small file named \texttt{README} explaining how you solved the exercise.

The error cases are:

\begin{itemize}
	\item If there are no argument
\end{itemize}

\subsection{Example}

\begin{lstlisting}[language=bash]
$./longest_words.py bonjour ahbon
bonjour
$./longest_words.py abc def ghi
abc,def,ghi
$./longest_words.py baobab bonnet arbre
baobab,bonnet
$./longest_words.py 
Usage: ./longest_words.py arg1 [arg2]...
$./longest_words.py q w e r t y
q,w,e,r,t,y
\end{lstlisting}

\section{Evaluation}
\subsection{Instructions}
\newdate{duedate}{30}{01}{2019}

You have to send a tarball (a \texttt{.tar}, not a \texttt{.zip} nor a \texttt{.rar}) of your work before \displaydate{duedate}, 23h59 at nikointhehood@gmail.com.

The subject and the content of your email are not important but your tarball should follow the format \texttt{firstname\_lastname.tar.bz2} (of course, you have to replace \texttt{firstname} and \texttt{lastname} by yours).

The command to work with tarballs is \texttt{tar}. To create one, use the following command:

\begin{lstlisting}[language=bash]
$ls
subject/ tutorial/
# c stands for create, v stands for verbose, f allows one to input a custom name for the archive and j indicates the usage of bzip2
$tar cvfj firstname_lastname.tar.bz2 subject/
(...)
$ls
subject/ tutorial firstname_lastname.tar.bz2
\end{lstlisting}

To verify that your archive contains everything you want, copy it somewhere else and decompress it:

\begin{lstlisting}[language=bash]
$ls
subject/ tutorial firstname_lastname.tar.bz2
$mkdir /tmp/test
$cp firstname_lastname.tar.bz2 /tmp/test
$cd /tmp/test
# x stands for eXtract, v for verbose and f allows one to indicate which file tar should work with
$tar xvf firstname_lastname.tar.bz2
$ls -l
# You should see what I will obtain when extracting your archive
(...)
\end{lstlisting}

\subsection{A word on cheating}

You have to remember that you should be studying for your own good. Cheating will not bring you any good in the long term; it is fine not to be able to finish every exercise of the subject, your main goal is to train and learn things.

Any form of cheating will immediately bring your grade down to 0. Additionally, your main teacher will be taken notice of that.

\textit{Note:} Changing the name of some variables will of course not trick the anti-cheat engine :)

\end{document}